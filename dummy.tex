%!TEX TS-program = xelatex
%!TEX encoding = UTF-8 Unicode
\documentclass[10pt]{report}
\usepackage{xltxtra,fontspec,xunicode}
\defaultfontfeatures{Scale=MatchLowercase}
%\setromanfont[Numbers=Uppercase]{Dosis}
%\setmonofont[Scale=0.90,Ligatures=NoCommon]{Courier}
\usepackage{amsmath, amscd}
\usepackage{draftwatermark}
%\usepackage{amssymb}
\usepackage[final]{pdfpages}
\usepackage{hyperref}
\usepackage{verbatim}
\usepackage{eurosym}
\usepackage{booktabs}
\usepackage{SIunits}
\usepackage{wasysym}
\usepackage{listings}
%\usepackage[detect-weight=true,load-configurations=binary]{siunitx}
%\usepackage{apacite}
\hypersetup{
    colorlinks,
    citecolor=black,
    filecolor=black,
    linkcolor=black,
    urlcolor=black
}
%\renewcommand{\refname}{Literatuurlijst}
%\usepackage{gnuplottex}
\usepackage{graphicx, color}
% laprint prerequisites
\usepackage{graphicx, color, psfrag}
% set lingual properties
%\usepackage[dutch]{babel}
\usepackage[english]{babel}
%\usepackage[dutch=nohyphenation]{hyphsubst}
\usepackage[toc]{glossaries}
\usepackage{epstopdf}
\usepackage{tikz}
\usepackage[utf8]{inputenc}
\usepackage{import}
\usepackage{algorithm}
\usepackage{algorithmicx}
\usepackage{algpseudocode}
%\usepackage[none]{hyphenat}
\newif\ifpdf
\ifx\pdfoutput\undefined
   \pdffalse
\else
   \pdfoutput=1
   \pdftrue \fi
\ifpdf
   \usepackage{graphicx}
   \usepackage{epstopdf}
   \DeclareGraphicsRule{.eps}{pdf}{.pdf}{`epstopdf #1}
   \pdfcompresslevel=9
\else
   \usepackage{graphicx}
\fi
% IMPORTING INKSCAPE
\newcommand{\executeiffilenewer}[3]{%
\ifnum\pdfstrcmp{\pdffilemoddate{#1}}%
{\pdffilemoddate{#2}}>0%
{\immediate\write18{#3}}\fi%
}
\newcommand{\includesvg}[1]{%
\executeiffilenewer{#1.svg}{#1.pdf}%
{inkscape -z -D --file=#1.svg %
--export-pdf=#1.pdf --export-latex}%
\input{#1.pdf_tex}%
}
\graphicspath{{graphics/}}

%\usepackage{../.TeX/packages/reqs}

\newcommand{\client}{NeedMe* }
\newcommand{\product}{Key*\ }

\begin{document}
\SetWatermarkText{Confidential}
\author{David Asabina\\\\
  Engineer, Suprnovae \\
  \url{mailto:david@suprnovae.com}\\
  Rotterdam, Nederland
}
%\date{\today}
\title{\textbf{Design}\\{KeyTag}}
\maketitle
\thispagestyle{empty}

%\clearpage
%\pagenumbering{roman}
%\section*{Voorwoord}
%\label{Voorwoord}
%
%\clearpage
%\setcounter{page}{1}
%%\setcounter{section}{1}
%%\setcounter{subsection}{0}
%\tableofcontents
%\clearpage
%\listoffigures
%%\listoftables

\clearpage
\pagenumbering{arabic}
\section{Introduction}
\label{Introduction}
Suprnovae is developing a product based on a concept proposed by Gerard Bakker \url{mailto:gerard@needme.nl}
from NeedMe on February 21, 2012. The concept regards a product which assists the owner in locating a paired 
device. The product will inversely allow the owner the possibility being located by utilizing 
the feature as implemented on the paired device. 

Besides the locating assist functionality, the product should offer the owner a minimal USB mass 
storage capacity of 16 GB and be compact enough to function as a keyfob.

\clearpage
\section{Hardware}
\subsection{Energy Consumption}
A small energy usage profile will be necessary to render the product a viable candidate for the modern
consumer electronics market. 

%Assuming that current consumption during moments of moderate MCU an RF activity
%amounts to \SIunits{30}{\milli\ampere}.
For each battery the current supllyable for a entire year could be calculated 
by solving expression \ref{Icont}.
\begin{equation}
	\label{Icont}
	I_{continuous} \times t_{year} = C \times \frac{1}{24} \times \frac{1}{364.25}
\end{equation}

In order to minimize the current usage over time, the device can be designed 
to operate with a fixed dutycycle ($D$), periodically activating the radio 
and necessary peripherals for brief periods of time ($\tau$) and sleeping for 
the remaining time of the respective period.
\begin{equation}
	D = \frac{\tau}{T}
\end{equation}

The continuous current consumable within a year period could be expressed in
relation to the drawn current during activity ($I_{active}$) and the drawn 
current while in sleep mode ($I_{sleep}$). In order to guarantee a single-charge
life-time of at least a full calendar year, the product between the system's duty 
cycle and current usage should not exceed the calculated continuous current to
be drawn in a year (see expression \ref{dcyear}).
\begin{equation}
	\label{dcyear}
	I_{cont} \times t_{year} \geq \frac{\tau}{T}\cdot I_{active} + \frac{T - \tau}{T}\cdot I_{sleep}
\end{equation}

\subsubsection{Energy Supplies}
Form-factor, chemical construction and electrical properties of a selection of
batteries are portrayed in table \ref{batterycomp} wherever this information 
has been available. 
%The requirements in relation to the form-factor of the product
%demand a solution 

\begin{table}[h!]
	\begin{center}
		\caption{Comparison of readily accessible battery-cells for mass production}
		\label{batterycomp}
		\begin{tabular}{lllrll}
			\toprule
			Name & Chemical & Electrical & Price & Rechargeable & Form Factor \\
			\midrule
			\href{http://www.google.com}{Varta V80H} & NiMh & \SIunits{80}{\milli\ampere\hour} @ \SIunits{1.2}{\volt} & \euro{3.25} & no & - \\
			Sanyo 18650 Enerpower & Li-Ion & \SIunits{2600}{\milli\ampere\hour} @ \SIunits{3.7}{\volt} & \euro{14.50} & yes & - \\ % 
			\href{http://www.alibaba.com/product-gs/523114211/li_polymer_rechargeable_battery.html}{SABC 703048} & Li-Polymer & \SIunits{1000}{\milli\ampere\hour} @ \SIunits{3.7}{\volt} & - & yes & \SIunits{7.3}{\milli\meter}x\SIunits{30}{\milli\meter}x\SIunits{48}{\milli\meter} \\ % http://www.alibaba.com/product-gs/523114211/li_polymer_rechargeable_battery.html/
			\href{http://www.alibaba.com/product-gs/458961353/VERY_useful_High_power_LIR3048_Rechargeable.html}{Leson LIR3048} & Li-Ion & \SIunits{300}{\milli\ampere\hour} @ \SIunits{3.6}{\volt} & - & yes & \diameter \SIunits{30.5}{\milli\meter}x\SIunits{4.8}{\milli\meter} \\
			\href{http://www.alibaba.com/product-gs/464442688/3_7V_Rechargeable_battery.html}{ZERNE 853440} & Li-Polymer & \SIunits{1100}{\milli\ampere} @ \SIunits{3.7}{\volt} & - & yes & \SIunits{8.5}{\milli\meter}x\SIunits{34}{\milli\meter}x\SIunits{40}{\milli\meter} \\
			\href{http://www.alibaba.com/product-gs/464442688/3_7V_Rechargeable_battery.html}{XRHD XRHD453040} & Li-Polymer & \SIunits{520}{\milli\ampere\hour} @ \SIunits{3.7}{\volt} & - & yes & \SIunits{5.0}{\milli\meter}x\SIunits{52}{\milli\meter}x\SIunits{76}{\milli\meter} \\
			\href{http://www.alibaba.com/product-gs/558977449/3_7V_rechargeable_battery_652020.html}{JK 652020} & Li-Ion & \SIunits{180}{\milli\ampere} @ \SIunits{3.7}{\volt} & - & yes & - \\
			\bottomrule
		\end{tabular}
	\end{center}
\end{table}
\clearpage

\subsubsection{Energy Demand}
Table \ref{currentCC2571RF} and \ref{currentCC2571PowerModes} contain current 
consumption data extracted from the datasheet of the CC2571 controller 
\cite[p.4]{cc2541datasheet}. 

During active operation with radio activity total current usage would be
$<$ \SIunits{40}{\milli\ampere}.

%Considering the fact that \client requires the device to be capable of operating
%on a single charge in relation to the available battery-cell options available
%leaves designers no other option but to operate the device at fixed duty-cycles.

\begin{table}
	\begin{center}
		\caption{Current consumption during RF activity}
		\label{currentCC2571RF}
		\begin{tabular}{ll}
			\toprule
			Condition & Current Usage \\
			\midrule
			RX standard mode + low MCU activity & \SIunits{17.9}{\milli\ampere} \\
			RX high-gain mode + low MCU activity & \SIunits{20.2}{\milli\ampere} \\
			TX \SIunits{-20}{\deci\bel\meter} mode + low MCU activity & \SIunits{16.8}{\milli\ampere} \\
			TX \SIunits{0}{\deci\bel\meter} mode + low MCU activity & \SIunits{18.2}{\milli\ampere} \\
			\bottomrule
		\end{tabular}
	\end{center}
\end{table}
\begin{table}
	\begin{center}
		\caption{Current consumption of core during the Power Modes}
		\label{currentCC2571PowerModes}
		\begin{tabular}{ll}
			\toprule
			Condition & Current Usage \\
			\midrule
			Power mode 1: Digital regulator + \SIunits{32.768}{\kilo\hertz} XOSC + POR + TMR + RAM retention + flash retention & \SIunits{270}{\micro\ampere} \\
			Power mode 2: \SIunits{32.768}{\kilo\hertz} XOSC + POR + TMR + RAM retention + flash retention & \SIunits{1}{\micro\ampere} \\
			Power mode 3: POR + RAM retention + flash retention & \SIunits{0.5}{\micro\ampere} \\
			\SIunits{32.768}{\kilo\hertz} XOSC + limited flash & \SIunits{6.7}{\milli\ampere} \\
			\bottomrule
		\end{tabular}
	\end{center}
\end{table}

\begin{table}
	\begin{center}
		\caption{Estimated current consumption of hardware supplementary to core}
		\label{currentsupest}
		\begin{tabular}{ll}
			\toprule
			Peripheral & Current Usage \\
			\midrule
			Piezo element & \SIunits{7.5}{\milli\ampere} @ \SIunits{3.3}{\volt} \\
			LED & \SIunits{15}{\milli\ampere} @ \SIunits{3.3}{\volt} \\
			\bottomrule
		\end{tabular}
	\end{center}
\end{table}
\begin{table}
	\begin{center}
		\caption{Estimated leakage current for hardware supplementary to core}
		\label{currentsupleak}
		\begin{tabular}{ll}
			\toprule
			Peripheral & Current Usage \\
			\midrule
			Piezo leakage & \SIunits{1}{\milli\ampere} @ \SIunits{3.3}{\volt} \\
			LED leakage & \SIunits{1}{\milli\ampere} @ \SIunits{3.3}{\volt} \\
			Other & \SIunits{1}{\milli\ampere} @ \SIunits{3.3}{\volt} \\
			\bottomrule
		\end{tabular}
	\end{center}
\end{table}

\subsection{Subsystems}
\subsubsection{Piezo Circuit}
In order to generate sound a pi\"ezo element has been selected. These elements
enable the generate of sharp audio signals at a relatively low energy cost.

The Mallory Sonalert AST1240MLQ has been the only element in stock for delivery
by Mouser during the design phase. All other options where either, too 
expensive or didn't comply to the form-factor requirements. The AST1240MLQ is
rated for operation at \SIunits{3}{\volt} peak-peak with maximum current
consumption of \SIunits{5}{\milli\ampere}\cite{piezodatasheet}.

The pi\"ezoelectric component will be shunted by a diode with a reverse 
breakdown voltage of \SIunits{5}{\volt} to guard against the development 
of unwanted voltage levels, may any stresses be exerted on the component.

The CC2541 contains 21 \SIunits{4}{\milli\ampere} and 
2 \SIunits{20}{\milli\ampere} IO pins. As a safety measure the entire 
pi\"ezoelectric circuit should be designed considering a driving pin 
supporting merely \SIunits{4}{\milli\ampere}.

\begin{figure}[hbt]
	\input{circuits/quick}
	\centerline{\box\graph}
	\caption{Customized caption}
	\label{symlabelforquick}
\end{figure}
\begin{figure}[hbt]
	\input{circuits/earphones}
	\centerline{\box\graph}
	\caption{Earphone caption}
	\label{symlabelforearphone}
\end{figure}
\begin{figure}[hbt]
	\input{circuits/test}
	\centerline{\box\graph}
	\caption{My first CM figure}
	\label{symlabelforme}
\end{figure}
%A MOSFET requires a pull-up to determine the MOSFET idle state. The advantage
%of the FET is that the entire driving circuit is purely voltage based.
%
A NPN BJT drives the pi\"ezoelectric element. The $\beta$ of the selected
component varies between $200$ and $250$.
%\bibliographystyle{apacite}
%\bibliography{sources}
\bibliography{../sources/sources}
%\addcontentsline{toc}{section}{Referenties}
\bibliographystyle{amsplain}

%\clearpage
%\renewcommand \thesection{\Roman{section}}
%%\renewcommand \thesubsection{\roman{section}.\roman{subsection}.\roman{subsubsection}}
%\addcontentsline{toc}{section}{Bijlages}
%\setcounter{page}{1}
%\setcounter{section}{1}
%\setcounter{subsection}{0}
%\includepdf[pages={1},landscape=true,lastpage=1,addtotoc={1,subsubsection,1,Title,schema-dig1}]{file.pdf}
\end{document}
